\ChapterS{Заключение}

\epigraph{Горе тому, кто читает только одну книгу.}{Дж. Герберт}

Первая часть курса лекций была посвящена в значительной степени
теоретическим основам программирования — базовым определениям,
принципам функционирования компьютеров, подходам к написанию и
выполнению программ. Несмотря на то, что первая часть, в принципе,
содержит весь необходимый материал для написания любой
программы\footnote{Напомним, по теореме Бёма — Якопини любую программу
можно написать только с помощью последовательного исполнения команд и
цикла while.}, этого на практике недостаточно. В больших программах на
первое место выходят проектирование, структуры данных и технологии.

Во второй части будут обсуждаться вопросы, связанные с написанием
программ, использующих объектно-ориентированную парадигму. Будут также
рассмотрены базовые структуры данных — массивы, списки, словари и
другие коллекции.

Все эти темы пригодятся в дальнейшем как в предстоящей
профессиональной деятельности (мало какая программа обходится без
использования сложных структур данных), так и при изучении других
дисциплин — «Построения и анализа алгоритмов», «Параллельного
программирования», «Объектно-ориентированного анализа и
проектирования» и так далее.
