\chapter{Управление выполнением}

\section{Операции и операторы}

\Par{Выражения и операции}

Одним из ключевых понятий в программировании является понятие выражения.

\begin{defn}
  \Term{Выражение}{Выражение} — это фрагмент программы, результатом
  вычисления которого является некоторое значение.
\end{defn}

Выражение, фактически, соответствует экземпляру какого-то типа. В этом
и заключается отличие выражений от других конструкция языка
программирования.

В некоторых случаях говорят о выражениях, которые не имеют
значений. Например, команду вывода на экран можно считать примером
такого выражения.

У выражения могут быть так называемые \Term{Побочные эффекты}{побочные
  эффекты}, когда сам процесс вычисления приводит к изменениям во
внешней среде. Например, к изменению состояния памяти.

Простейшее выражение — это литерал или переменная. В последнем случае
значением выражения будет объект, хранящийся в переменной. Применяя
операции можно комбинировать выражения, получая новые.

\begin{defn}
  \Term{Операция}{Операция} — это запись действий над одним или
  несколькими значениями (\Term{Операнд}{операндами}).
\end{defn}


В зависимости от количества операндов можно разделить операции на
\Term{Операция!унарная}{унарные}, \Term{Операция!бинарная}{бинарные} и
\Term{Операция!унарная}{тернарные} — один, два и три операнда
соответственно. Например, операция сложения — бинарная, а операция
транспонирования матрицы (если она есть в языке) — унарная.

Если знак операции, ставится перед операндами, то говорят о
\Term{Нотация!префиксная}{префиксной записи} или нотации. Если после,
то о
\Term{Нотация!постфиксная}{постфиксной}. \Term{Нотация!инфиксная}{Инфиксная
  нотация} (которая используется в арифметике) подразумевает
использование знака операции между операндами. Например, $2 + 3$ — это
инфиксная запись сложения, а $+ 2 3$ — префиксная. Несмотря на то, что
инфиксная запись наиболее привычна человеку, она, возможно, является
наименее удобной для вычислений.

\Par{Операции языка C\#}

\Par{Операторы}

\Par{Условный оператор}

\begin{lstlisting}

\end{lstlisting}

\begin{example}
  Рассмотрим фрагмент программы, помещающей в переменную $y$ значение
  выражения $|x|$. По определению модуля

\begin{lstlisting}
if (x >= 0)
    y = x;
else
    y = -x;
\end{lstlisting}
\end{example}

\Par{Оператор выбора}

\Par{Цикл с предусловием}

\Par{Цикл с постусловием}

\Par{Цикл с параметром}

\Par{Цикл «для каждого»}

\Par{Операторы перехода}



\section{Подпрограммы и функции}

\Par{Понятие подпрограммы}

\Par{Декомпозиция задачи}

\Par{Лямбда-функции}

\Par{Вызов функции и возврат значения}

\Par{Локальные переменные и аргументы}

\Par{Замыкания}

\Par{Тип функции}

\Par{Делегаты}

\Par{Статические методы как функции}

\Par{Рекурсия}

\section{Исключительные ситуации}

\Par{Понятие исключительной ситуации}

\Par{Примеры исключений}

\Par{Перехват исключения}

\Par{Описание исключений}
