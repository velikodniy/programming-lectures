\newlength{\UDCLength}
\setlength{\UDCLength}{\maxof{\widthof{УДК}}{\widthof{ББК}}}
\noindent\makebox[\UDCLength][l]{УДК} \UDC\\
\noindent\makebox[\UDCLength][l]{ББК} \BBK\\
\noindent\makebox[\UDCLength][l]{} \AS

\vfill{}

\noindent
    {\addfontfeature{LetterSpace=30}Рецензенты}:

\noindent\textbf{Коровай~А.~В.}, к.~ф.-м.~н., зав. каф. ИиВТ ПГУ им. Т.~Г.~Шевченко

\noindent\textbf{Марков~Д.~А.}, ст.~преп. каф. НОиКР ПГУ им.~Т.~Г.~Шевченко

\vfill{}

\newlength{\ASShiftLength}
\setlength{\ASShiftLength}{-0.5em-\widthof{\AS}}
\noindent {\hspace{\ASShiftLength}\AS}\hfill%
\begin{minipage}[t]{1\columnwidth}%
  \noindent\hspace{2em}\textbf{\AuthorI}

  \noindent\hspace{2em}\Title. \SubTitle: \PubType.
  — Тирасполь,~\Year. — \pageref{LASTPAGE}~с. %%% FIXME Количество страниц
  \medskip{}

  \hspace{2em}{\small Курс состоит из нескольких разделов, в которых
    описываются основные понятия, связанные с технологией
    программирования, написанием программ и их выполнением, подробно
    рассматривается объектно-ориентированная парадигма
    программирования. Также кратко описаны распространённые алгоритмы
    и структуры данных. Изложение ведётся на примере языка
    программирования C\#.}

  \hspace{2em}{\small Предназначено для студентов, обучающихся по
    направлениям «Прикладная математика» и «Прикладная математика и
    информатика».}
\end{minipage}

\noindent
\begin{flushright}
  \begin{minipage}[t]{0.5\columnwidth}
    \noindent\makebox[\UDCLength][l]{УДК} \UDC\\
    \noindent\makebox[\UDCLength][l]{ББК} \BBK
  \end{minipage}
\end{flushright}

\vfill{}

\begin{center}
  Утверждено Научно-методическим советом ПГУ~им.~Т.~Г.~Шевченко
\end{center}

\vfill{}

\noindent {\small © \AuthorI, \Year.}

\thispagestyle{empty}
\newpage
