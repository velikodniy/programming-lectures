\chapter{Хранение данных}

\section{Типы данных и литералы}

\Par{Тип данных}

Одним из важнейших понятий в программировании является понятие типа
данных.

\begin{defn}
  \Term{Тип данных}{Тип данных} — это совокупность, включающая:
  \begin{itemize}
    \item множество допустимых значений какого-либо набора данных,
    \item множество допустимых над ним операций,
    \item смысл данных,
    \item способ их представления в памяти компьютера.
  \end{itemize}
\end{defn}

Любые данные, хранящиеся в памяти компьютера, так или иначе относятся
к какому-либо типу.

Разнородные данные требуют разных объёмов памяти для хранения,
обрабатываются различным образом. Например, целые и вещественные числа
кодируются в процессорах архитектуры IA-32 различным образом. Поэтому
для их обработки требуются различные машинные команды, даже если эти
числа равны.

Информация о том, данные какого типа расположены в памяти, может
помочь компилятору сгенерировать более эффективный машинный код. В
связи с этим, во многих распространённых языках программирования в
явном виде присутствует концепция типов.

\Par{Простые и сложные типы}

Из-за огромного разнообразия данных, которые требуется обрабатывать,
непосредственная поддержка каждого из них на аппаратном уровне,
очевидно, невозможна. Поэтому типы данных делят на две категории.

\begin{itemize}
\item \Term{Тип данных!простой}{Простые типы} не имеющие внутренней
  структуры, доступной программисту. Обычно поддерживаются
  непосредственно конкретной программно-аппаратной платформой.

  Например, числа обычно относят к простым типам.

\item \Term{Тип данных!сложный}{Сложные типы}, которые конструируются
  из базовых.

  Например, вектор можно представить как последовательность чисел, то
  есть он обладает внутренней структурой.
\end{itemize}

Работа программиста состоит во многом в описании данных,
присутствующих в стоящей задаче. Для этого необходимо описать
соответствующие сложные типы, которые конструируются из простых
средствами языка програмирования.

\Par{Базовые типы}

Программно-аппаратные платформы и языки программирования предоставляют
программисту некоторый набор предопределённых \Term{Тип
  данных!базовый}{базовых (примитивных) типов}.

Например, базовыми типами для процессора архитектуры AMD64 являются
1-, 2-, 4- и 8-байтовые целые числа, 10-байтовые числа с плавающей
точкой. Хотя на самом деле для кодирования любых данных, встречающихся
на практике, достаточно было бы только этих типов, для удобства многие
другие типы в высокоуровневых языках программирования также относят к
базовым.

В частности, в языке C\# присутствует ограниченный набор базовых
типов, которые приведены в таблице~\ref{tab:predefined-types}.

\begin{table}
  \begin{centering}
    \begin{tabular}{|l|l|}
      \hline
      Имя           & Описание\\
      \hline
      \hline
      \Lst{byte}    & 1-байтовое целое число без знака\\
      \Lst{sbyte}   & 1-байтовое целое число со знаком\\
      \Lst{short}   & 2-байтовое целое число со знаком\\
      \Lst{ushort}  & 2-байтовое целое число без знака\\
      \Lst{int}     & 4-байтовое целое число со знаком\\
      \Lst{uint}    & 4-байтовое целое число без знака\\
      \Lst{long}    & 8-байтовое целое число со знаком\\
      \Lst{ulong}   & 8-байтовое целое число без знака\\
      \hline
      \Lst{float}   & Действительное число одинарной точности\\
      \Lst{double}  & Действительное число двойной точности\\
      \Lst{decimal} & Действительное число с фиксированной запятой\\
      \hline
      \Lst{bool}    & Логическое значение\\
      \hline
      \Lst{char}    & Одиночный символ в кодировке Юникод\\
      \Lst{string}  & Строка (последовательность символов)\\
      \hline
      \Lst{object}  & Тип, являющийся основой для остальных типов\\
      \hline
    \end{tabular}\par
  \end{centering}
  
  \caption{Базовые типы языка C\#\label{tab:predefined-types}}
\end{table}

\Par{Классификация типов языка C\#}

Базовые типы языка C\# можно условно разделить на следующие группы.

\begin{itemize}
\item \Term{Тип данных!целочисленный}{Целочисленные типы},
  соответствующие различным диапазонам целых чисел.
\item \Term{Тип данных!вещественный}{Вещественные типы}, позоляющие
  приближённо описывать действительные числа с различной точностью.
\item \Term{Тип данных!логический}{Логический тип}.
\item \Term{Тип данных!символьный}{Символьный} и \Term{Тип
  данных!строковый}{строковый типы}, используемые для хранения
  фрагментов текста.
\item \Term{Тип данных!object}{Тип \Lst{object}} — вершина иерархии
  типов, не используется непосредственно.
\end{itemize}

Надо заметить, что эта классификация несколько отличается от
классификации, приведённой в стандарте на язык C\#.

\Par{Целочисленные типы}

Целочисленные типы составляют наиболее обширный класс базовых
типов. Это связано с тем, что целые числа — основной тип данных,
обрабатываемых процессором. Более того процессоры многих архитектур (в
основном, микроконтроллеры) поддерживают только целые числа.

Целые числа в памяти компьютера обычно занимают 1, 2, 4 или
8~байт. Также они могут рассматриваться как неотрицательные числа или
числа со знаком. Таким образом, получаем 8 целочисленных типов,
приведённых в таблице~\ref{tab:integral-types}.

\begin{table}
  \begin{centering}
    \begin{tabular}{|l|c|}
      \hline
      Имя           & Диапазон\\
      \hline
      \hline
      \Lst{byte}    & $-128\ldots127$\\
      \Lst{sbyte}   & $0\ldots255$\\
      \Lst{short}   & $-32768\ldots32767$\\
      \Lst{ushort}  & $0\ldots65535$\\
      \Lst{int}     & $-2147483648\ldots2147483647$\\
      \Lst{uint}    & $0\ldots4294967295$\\
      \Lst{long}    & $-9223372036854775808\ldots9223372036854775807$\\
      \Lst{ulong}   & $0\ldots18446744073709551615$\\
      \hline
    \end{tabular}\par
  \end{centering}
  
  \caption{Диапазоны значений целочисленных типов\label{tab:integral-types}}
\end{table}

Следует помнить о диапазонах представимых целых чисел. Например, число
$-128$ можно хранить в переменной типа \Lst{byte}, но вычислить его
модуль и поместить в ту же переменную не получится.

\Par{Вещественные типы}

Реализация вещественных типов с плавающей запятой часто представлена на
аппаратном уровне. Существует стандарт IEEE~754-2008 описывающий
формат преставления вещественных чисел половинной, одинарной, двойной
и четверной точности. В языке C\# базовыми явзяются числа одинарной
(тип \Lst{float}) и двойной точности (\Lst{double}) — их достаточно
для подавляющего большинства приложений.

Кроме самих чисел из различных диапазонов стандарт предусматривает
возможность работы с положительной и отрицательной бесконечностями и
так называемым «не числом» (NaN — от англ. not a
number). Положительная бесконечность получается, например, при делении
положительного числа на ноль. Результатом вычисления $\frac00$ будет
NaN.

Рассмотренные вещественные типы хранят значения в приближённом
виде. Это связано с ограничениями на количество хранимых знаков
мантиссы и экспоненты. Поэтому не рекомендуется сравнивать между собой
вещественные числа.

Например, результат суммирования пятидесяти слагаемых, равных~$0{,}1$,
не будет равен~$5{,}0$ при использовании чисел с плавающей
точкой. Дело в том, что число~$0{,}1$ непредставимо в двоичной системе
в виде конечной дроби. Часть значащих цифр отбрасывается, и
вместо~$0{,}1$ в вычислениях участвует другое число, хоть и
отличающееся на чрезвычайно малую величину.

Если необходима сохранность каждой значащей цифры, можно использовать
тип \Lst{decimal}. Однако он менее эффективен, так как реализован не
аппаратно, а программно.

Перечисленные типы, дапазоны их допустимых значений и количество
хранимых значащих цифр мантиссы приведены в
таблице~\ref{tab:real-types}.

\begin{table}
  \begin{centering}
    \begin{tabular}{|l|c|c|}
      \hline
      Имя            & Диапазон & Точность, цифр\\
      \hline
      \hline
      \Lst{float}    &
      $\pm1{,}5\times10^{-45}\ldots\pm3{,}4\times10^{38}\cup\{0\}$   &
      7\\
      \Lst{double}   &
      $\pm5{,}0\times10^{-324}\ldots\pm1{,}7\times10^{308}\cup\{0\}$ &
      15--16\\
      \Lst{decimal}  &
      $\pm1{,}0\times10^{-28}\ldots\pm7{,}9\times10^{28}\cup\{0\}$   &
      28\\
      \hline
    \end{tabular}\par
  \end{centering}
  
  \caption{Диапазоны значений типов, описывающих действительные
    числа\label{tab:real-types}}
\end{table}

\Par{Логический тип}

Логический тип данных \Lst{bool} включает лишь два значения — истину и
ложь. Он может быть полезен для хранения информации об объектах,
имеющих только два состояния. Удобство его заключается также в том,
что в C\# присутствуют логические операции.

\Par{Символьный тип}

Символьный тип \Lst{char} преназначен для хранения отдельных символов —
букв, цифр, знаков препинания и других. Символ может быть любым,
предусмотренным стандартом Юникод. Это позволяет хранить в памяти
буквы практически любого алфавита, иероглифы и специальные знаки.

Также к символам относят так называемые \Term{Управляющая
  последовательность}{управляющие поледовательности} — символы,
которые не отображаются на экране, но выполняют какие-либо
действия. Например, к ним относятся символ перехода на новую строку,
символ табуляции и так далее.

\Par{Строковый тип}

Для хранения фрагментов текста длиннее одного символа используется так
называемый строковый тип \Lst{string}. Он предназначен для хранения
\Term{Строка}{строк} — неизменяемых последовательностей символов
Юникода.

\Par{Тип \Lst{object}}

Многие языки программирования (например, Haskell) имеют развитую
систему типов, которые образуют иерархию. Как правило иерархия
повторяет вложенность соответствующих множеств в математике. Например,
целые числа являются частным случаем вещественных чисел, что может
быть отражено в языке программирования.

В языке C\# вершиной иерархии типов является тип \Lst{object}.  Сам
по себе для хранения данных он не используется, но может быть полезен,
если конкретный тип данных неизвестен.

\Par{Литералы}

Для того, чтобы работать с данными, должен существовать способ их
описания. Для этого используются литералы.

\begin{defn}
  \Term{Литерал}{Литерал} — неименованная константа какого-либо типа
  данных.
\end{defn}

Например, число $15$ — это неименованная целочисленная константа, то
есть целочисленный литерал.

В языке C\# можно выделить следующие виды литералов:
\begin{enumerate}
\item целочисленные;
\item вещественные;
\item символьные;
\item строковые;
\item логические;
\item литерал null.
\end{enumerate}

Примеры литералов приведены в таблице~\ref{tab:literal-example}.

\begin{table}
  \begin{centering}
    \begin{tabular}{|c|c|c|}
      \hline 
      Литерал & Значение & Тип \\
      \hline 
      \hline 
      \Lst{+12} & $12$ & \Lst{int}\\
      \hline 
      \Lst{12U} & $12$ & \Lst{uint}\\
      \hline 
      \Lst{-12L} & $-12$ & \Lst{long}\\
      \hline 
      \Lst{-12UL} & $-12$ & \Lst{ulong}\\
      \hline 
      \Lst{0x12} & $12_{16}=18_{10}$ & \Lst{int}\\
      \hline 
      \Lst{0x1BUL} & $1\mathrm{B}_{16}=27_{10}$ & \Lst{ulong}\\
      \hline 
      \Lst{12.3} & $12{,}3$ & \Lst{double}\\
      \hline 
      \Lst{12.3F} & $12{,}3$ & \Lst{float}\\
      \hline 
      \Lst{-12.3E2} & $-12{,}3 \times 10^2 = -1230$ & \Lst{double}\\
      \hline 
      \Lst{1.2E-2F} & $1{,}2 \times 10^{-2} = 0{,}012$ & \Lst{float}\\
      \hline 
      \Lst{'A'} & «A» & \Lst{char}\\
      \hline 
      \Lst{"Abc"} & «Abc» & \Lst{string}\\
      \hline 
      \Lst{'\textbackslash u004A'} & «J» & \Lst{char}\\
      \hline 
      \Lst{"\textbackslash u03BE"} & «$\xi$» & \Lst{string}\\
      \hline 
    \end{tabular}\par
  \end{centering}

  \caption{Примеры литералов\label{tab:literal-example}}
\end{table}

Рассмотрим каждый из видов на примерах.

\Par{Целочисленные литералы}

\Term{Литерал!целочисленный}{Целочисленные литералы} представляют
собой последовательность десятичных цифр, перед которой может стоять
знак «\Lst{+}» либо «\Lst{-}». По умолчанию литералы относятся к
типу \Lst{int}, но можно добавить к числу суффикс «\Lst{L}» для
указания на принадлежность к типу \Lst{long}. Также можно
использовать суффикс «\Lst{U}» для указания на то, что
целочисленный тип беззнаковый. (Для написания суффиксов лучше
использовать заглавные буквы: например, буква L и цифра 1 в некоторых
шрифтах очень похожи и могут привести к путанице.)

Также существует возможность использовать шестнадцатеричную форму
записи чисел. В этом случае необходимо добавить перед литералом
«\Lst{0x}».

\Par{Вещественные литералы}

Особенностью \Term{Литерал!вещественный}{вещественных литералов}
является использование точки для отделения дробной части числа. По
умолчанию, вещественные литералы принадлежат типу \Lst{double}. Для
указания на принадлежность к типу \Lst{float} необходимо добавить
суффикс «\Lst{F}», к типу \Lst{decimal} — суффикс «\Lst{M}».

Для записи очень больших и очень малых чисел можно использовать
специальную нотацию вида: «\Lst{\uline{число}E\uline{порядок}}».
Здесь \Lst{\uline{число}} — вещественный литерал, а
\Lst{\uline{порядок}} — целочисленный. Эта запись соответствует
выражению
\[\textrm{\Lst{\uline{число}}} \times 10^{\textrm{\Lst{\uline{порядок}}}}.\]

\Par{Символьные литералы}

\Term{Литерал!символьный}{Символьные литералы} представляют собой
отдельные символы, заключённые в одинарные кавычки. Тип символьных
литералов — \Lst{char}.

Для записи самого символа одинарной кавычки и некоторых других
символов применяют экранирование — запись последовательности,
начинающейся с символа «\Lst{\textbackslash{}}» и соответствующей
отдельному специальному символу. Допустимые последовательности
приведены в таблице \ref{tab:escseq}.

\begin{table}
  \begin{centering}
    \begin{tabular}{|c|l|}
      \hline 
      Последовательность & Символ\\
      \hline 
      \hline 
      \Lst{\textbackslash '} & Одинарная кавычка («'»)\\
      \hline 
      \Lst{\textbackslash "} & Двойная кавычка («\textquotedbl»)\\
      \hline 
      \Lst{\textbackslash \textbackslash} & Обратная наклонная черта («\textbackslash»)\\
      \hline 
      \Lst{\textbackslash 0} & Символ с кодом 0 (признак конца строки)\\
      \hline 
      \Lst{\textbackslash a} & Звуковой сигнал\\
      \hline 
      \Lst{\textbackslash b} & Удаление последнего символа\\
      \hline 
      \Lst{\textbackslash f} & Переход на следующую страницу\\
      \hline 
      \Lst{\textbackslash n} & Переход в начало следующей строки\\
      \hline 
      \Lst{\textbackslash r} & Переход в начало текущей строки\\
      \hline 
      \Lst{\textbackslash t} & Переход к следующей позиции табуляции\\
      \hline 
      \Lst{\textbackslash v} & Переход на символ вниз\\
      \hline 
    \end{tabular}\par
  \end{centering}

  \caption{Специальные символы\label{tab:escseq}}
\end{table}

Также можно указать любой символ кодировки Юникод по его коду,
используя специальную последовательность вида
«\Lst{\textbackslash u\uline{код}}», где
\Lst{\uline{код}} — четыре шестнадцатеричные цифры кода символа.

\Par{Строковые литералы}

\Term{Литерал!строковый}{Строковые литералы} — это последовательности
из нуля и более символов, заключённые в двойные кавычки. Строка из
нуля символов называется \Term{Строка!пустая}{пустой
  строкой}. Строковые литералы относятся к типу \Lst{string}.

Если в строке встречаются символы обратной косой черты, то их
приходится удваивать, так как они имеют специальный смысл. При большом
их количестве это снижает читаемость текста. В таком случае можно
использовать \Term{Литерал!строковый!буквальный}{буквальные строковые
  литералы}. Они записываются аналогично, но перед первой кавычкой
ставится символ «@».

Например, следующие две строки эквивалентны:
\begin{itemize}
\item\Lst{"c:\textbackslash\textbackslash Documents\textbackslash\textbackslash
  text.txt"}
\item\Lst{@"c:\textbackslash Documents\textbackslash text.txt"}
\end{itemize}

Чтобы включить в состав такого литерала сам символ двойной кавычки,
его нужно удвоить.

Недостатком буквальных строковых литералов является то, что в них
нельзя использовать управляющие последовательности.

\Par{Логические литералы}

\Term{Литерал!логический}{Логические литералы} относятся к типу
\Lst{bool} и позволяют представить только одно из двух значений:
истина (литерал \Lst{true}) либо ложь (\Lst{false}).

\Par{Литерал \Lst{null}}

Иногда возникает необходимость сохранить в переменной значение, не
принадлежащее множеству допустимых значений её типа. В этом случае в
C\# можно использовать \Term{Литерал!null}{литерал }\Lst{null}.  Для
того, чтобы объект какого-либо типа мог хранить это значение,
необходимо использовать расширенный тип, имя которого состоит из имени
исходного типа и знака «?». Например, «\Lst{int?}».

Также литерал \Lst{null} используется в качестве некоторого «нулевого»
значения ссылки на область в памяти, не указывающего ни на один
существующий объект.

\Par{Производные типы}

В языке C\#, как и во многих других языках, присутствуют различные
средства создания производных типов данных. В частности, распространёнными
средствами являются:
\begin{itemize}
\item перечисления — средство группировки констант по какому-либо
  признаку;
\item массивы — наборы пронумерованных данных одного типа (например,
  тип, описывающий результаты многочисленных измерений);
\item структуры — типы данных, позволяющие группировать объекты
  различных типов (например, тип, описывающий паспортные данные);
\item классы — дальнейшее развитие структур, одно из основных понятий
  объектно-ориентированного программирования
\item и другие.
\end{itemize}

\Par{Обобщённые типы}

Производные типы могут быть обобщёнными. Например, тип данных,
соответствующий списку значений, можно обобщить, не указывая конкретный
тип хранимых элементов, а используя его в качестве параметра T
некоторого обобщённого типа «Список элементов типа T». Использование
обобщений уменьшает размер программы, так как позволяет лишь один раз
описывать схожие действия для различных родственных типов.

\Par{Преобразования типов}

Многие типы данных, используемые на практике, допускают взаимные
преобразования друг в друга. Так, например, величины типа \Lst{int}
могут при необходимости преобразовываться к типу
\Lst{double}. Подобные преобразования могут быть удобны, так как у
процессора архитектуры IA-32 нет команды для сложения целых и
вещественных чисел, только для сложения двух целых и двух
вещественных. Поэтому в выражении «1 + 1.0» оба аргумента должны быть
приведены к одному и тому же типу.

Преобразования типов могут быть как неявными, выполняемыми
автоматически компилятором, так и явными, по указанию программиста.

Неявные преобразования в некоторых случаях могут быть нежелательными,
и, кроме того, они усложняют контроль за типами данных в программе.  В
связи с этим, в некоторых языках неявные преобразования сведены к
минимуму или запрещены вовсе. Такие языки называются языками со
\Term{Типизация!строгая}{строгой типизацией}.

% переместить
Описание типов данных — один из важнейших этапов работы над
программой.  Удачно выбранная система типов, хорошо соответствующая
предметной области, в которой разрабатывается программа, позволяет
значительно упростить процесс программирования и избежать многих
ошибок.

\section{Переменные и константы}

\Par{Переменные}

Все данные, которые используются в ходе выполнения программы должны
располагаться в памяти компьютера. Для доступа к данным в первых
программах использовались адреса ячеек памяти, но подобный подход
достаточно неудобен. Для упрощения работы с данными была введена
абстракция, пришедшая из математики, — понятие переменной.

\begin{defn}
  \Term{Переменная}{Переменная} — это объект в памяти компьютера,
  хранящий значение какого-либо типа.
\end{defn}

\Par{Идентификаторы}

Для доступа к переменным, им присваиваются
\Term{Идентификатор}{идентификаторы} — имена, записываемые по
определённым правилам. Правила записи идентификаторов различаются в
различных языках. В некоторых из них разрешено использовать любые
последовательности практически любых символов кодировки Юникод, а в
некоторых — лишь единственную латинскую букву.

В языке C\# упрощённые правила записи идентификаторов выглядят
следующим образом. Идентификатор — это последовательность букв, цифр и
знаков подчёркивания, начинающаяся не с цифры. Вообще говоря, буквы
могут принадлежать любому алфавиту, описанному в Юникод, но
рекомендуется использовать только латинские буквы. Во-первых, это
упростит чтение и редактирование программы иноязычными разработчиками,
во-вторых, некоторые символы в различных алфавитах выглядят одинаково
(например латинская «o» и кириллическая «о»), что может вызвать
путаницу, и наконец, поддержка нелатинских букв в идентификаторах
может отсутствовать в компиляторе.

Регистр символов имеет значение. Поэтому, к примеру, переменные
«\Lst{a}» и «\Lst{A}» считаются различными.

Желательно давать переменным имена, отражающие суть хранимых в них
данных. Это значительно упрощает чтение программы. Например, для
переменной, хранящей количество значений какой-либо величины, имя
«\Lst{NumberOfValues}» является более предпочтительным, чем
«\Lst{x}».
% нельзя использовать ключевые слова языка

\Par{Виды типизации}

Любое значение, хранимое в переменной относится к какому-либо типу.
Существует несколько подходов к организации связи переменной и типа
данных в языке.

При \Term{Типизация!статическая}{статической типизации} каждая
переменная связывается с типом данных на этапе трансляции программы не
может изменять свой тип в дальнейшем. Если указать, что переменная
хранит целочисленные значения, то в неё в дальнейшем невозможно будет
поместить что-либо другое.  К статически типизированным языкам
относятся C++, Pascal, Java и другие.

В языках с \Term{Типизация!динамическая}{динамической типизацией} тип
переменной определяется во время выполнения программы в момент
присваивания ей значения. Это позволяет хранить в одной и той же
переменной значения различных типов.  Динамическая типизация
присутствует в таких языках, как Python, PHP, Ruby и т.~д.

Язык C\# использует смешанный подход. С одной стороны, на этапе
трансляции проводятся проверки типов как при статической типизации. С
другой стороны, если объявить переменную с использованием ключевого
слова \Lst{dynamic} вместо имени типа, то её тип будет определяться
динамически по хранимому значению. Однако, эта возможность на практике
используется сравнительно редко.

Статическая типизация имеет ряд преимуществ. В частности, она
позволяет уже на этапе компиляции программы обнаруживать ряд ошибок,
таких как присваивание значения не той переменной. Кроме того, в
отличие от динамической типизации, она не требует выполнения проверок
во время выполнения программы, что уменьшает время работы.

\Par{Объявление переменных}

В C\# для того, чтобы связать переменную с типом и выделить под неё
место в памяти, необходимо её
\Term{Объявление!переменной}{объявить}. Все переменные в программе
должны быть объявлены. Формат объявления:

\begin{center}
  \Lst{\uline{тип}}\Lst{ }\Lst{\uline{список переменных}}\Lst{;}
\par\end{center}

Здесь \Lst{\uline{список переменных}} — это перечень
идентификаторов через запятую.

\begin{example}
  Переменные \Lst{a} и \Lst{b} после приведённого ниже
  объявления будут иметь тип \Lst{int}, а переменная \Lst{c} —
  \Lst{double}.
\begin{lstlisting}
int a, b;
double c;
\end{lstlisting}
\end{example}

\Par{Инициализация}

Переменную при объявлении можно
\Term{Инициализация}{инициализировать}, то есть присвоить ей
какое-либо значение. для этого нужно после соответствующего
идентификатора добавить

\begin{center}
  \Lst{= }\Lst{\uline{значение}}
\par\end{center}

После знака «=» может стоять литерал или другая переменная, уже имеющая
значение.

\begin{example}
  Переменная \Lst{g} инициализируется значением $9{,}81,$ а
  переменные \Lst{x} и \Lst{y} не инициализированы.

\begin{lstlisting}
double x, g=9.81, y;
\end{lstlisting}
\end{example}

Если переменная инициализируется, то существует возможность не
указывать тип переменной, записав вместо него ключевое слово
\Lst{var}. В этом случае переменная получит тип, совпадающий с
типом присваиваемого значения. При этом сохраняется статическая
типизация, так как переменная связывается с некоторым типом, хоть он и
не указан явно.

\begin{example}
Переменная \Lst{w} получает тип \Lst{double}, так как к нему
относится литерал \Lst{12.3}.

\begin{lstlisting}
var w = 1.23;
\end{lstlisting}
\end{example}

После объявления переменной появляется возможность использовать
указанный идентификатор, однако не всегда объявление позволяет
утверждать, что память под объект выделена.

\Par{Значимые и ссылочные типы}

В языке C\# все типы данных делятся на \Term{Тип
  данных!значимый}{значимые} и \Term{Тип данных!ссылочный}{ссылочные}.
К значимым типам относятся целочисленные и вещественные типы,
логический тип, структуры и перечисления. Все остальные типы относятся
к ссылочным.

Переменная значимого типа хранит непосредственно значение указанного
типа, и память под это значение выделяется при
объявлении. Идентификатор переменной связывается с выделенным участком
памяти и по нему можно получить доступ к хранимому значению.

Переменная ссылочного типа хранит только ссылку на область памяти, где
располагается значение. При объявлении память выделяется только для
ссылки, а память для значения должна быть выделена отдельно. В языке
C\# для выделения памяти используется операция \Lst{new},
выделяющая область памяти под объект указанного типа и возвращающая
ссылку на неё.

\begin{example}
Переменная \Lst{x} — ссылка на объект типа
\Lst{System.Drawing.Bitmap(100, 100)}, соответствующего
изображениям размера $100\times100$ пикселей.  Переменная \Lst{y}
ссылается на тот же самый объект, так как была инициализирована
ссылкой \Lst{x}. В частности, отсюда следует, что изменение
\Lst{x} повлечёт за собой изменение \Lst{y}.

\begin{lstlisting}
var x = new System.Drawing.Bitmap(100, 100);
var y = x;
\end{lstlisting}
\end{example}

\Par{Сборка мусора}

Операции, позволяющей указать, что некоторая область памяти не
используется и её можно освободить, в C\# нет. Это связано с тем, что
виртуальная машина следит за ссылками на каждую выделенную область
памяти, и если ссылок на какую-то область больше нет, она
освобождается автоматически.  Этим занимается специальная подсистема,
называемая \emph{сборщиком мусора} (GC — garbage collector). Его
поддержка присутствует во многих языках программирования, таких, как
Python, Java, Lisp и т.~д.

Для повышения быстродействия разработчики языка могут отказаться от
сборщика мусора, но в этом случае возникает опасность появления
областей, помеченных как занятые, но не используемых, из-за ошибок
программиста, забывшего вставить команду на освобождение памяти. Такая
ситуация называется \Term{Утечка памяти}{утечкой памяти} (memory
leak).

\Par{Константы}

Если значение переменной не должно меняться во время выполнения
программы, то её можно объявить как константу, добавив перед
объявлением \Lst{const}.

\begin{example}
Константа $g=9{,}81.$

\begin{lstlisting}
const double g = 9.81;
\end{lstlisting}
\end{example}

Константы обязательно должны быть инициализированы, так как в
дальнейшем присвоить им какое-либо значение невозможно.

\section{Виды памяти}

\Par{Иерархия видов памяти}

Из-за технических ограничений, существующих в настоящее время, память
компьютера имеет разнородную структуру. Программы и данные, с которыми
работает компьютер, должны храниться в
\Term{Память!оперативная}{оперативной памяти}. Однако скорость доступа
к ней, как правило, меньше, чем требуется для эффективной работы
процессоров распространённых архитектур. Поэтому большинство машинных
команд предназначено для вычислений над данными, находящимися в особых
ячейках памяти, являющихся частью самого процессора — \Term{Регистр
  процессора}{регистрах}. Их количество ограничено, но доступ к ним
осуществляется за время, сравнимое со временем выполнения одной
команды.

Так как регистров мало, то для ускорения доступа к оперативной памяти,
часто используемые данные из неё копируются в \Term{Кэш
  процессора}{кэш процессора}.  Этот вид памяти работает быстрее, чем
оперативная память, но медленнее, чем регистры. С другой стороны, его
объём больше, чем объём данных, которые можно разместить в регистрах.

Оперативная память достаточно дорога и, как правило, требует
непрерывного электропитания, поэтому большие объёмы данных хранят во
внешней памяти — жёстких дисках, флеш-памяти, оптических дисках и
т.~д.

Таким образом, все рассмотренные виды памяти образуют следующую
иерархию:
\begin{enumerate}
\item регистры (суммарный объём обычно не превышает сотен байт);
\item кэш процессора (до нескольких мегабайт);
\item оперативная память (порядка нескольких гигабайт);
\item внешняя память (объём может значительно отличаться, порядка
  десятков гигабайт — нескольких терабайт).
\end{enumerate}

Этот список упорядочен по возрастанию объёма и уменьшению
быстродействия.

Разница во времени доступа к регистрам и внешней памяти огромна и
может отличаться на несколько порядков.

При написании программ обычно не требуется знать, где расположены
данные: в регистрах, в кэше или оперативной памяти. Компилятор сам
распределяет данные. При прикладном программировании можно условно
считать, что все объекты, с которыми работает программа, располагаются
в оперативной памяти. Впрочем, при написании программ, от которых
требуется высокая скорость расчётов, информация о размещении данных
становится важной.

\Par{Адресное пространство}

Выполнение программы начинается с загрузки её в память операционной
системой. Современные распространённые операционные системы
предоставляют каждому процессу собственное
\Term{Пространство!адресное}{адресное пространство}. При этом
размещение машинных команд и данных стандартизировано в рамках
операционной системы.

На рис.~\ref{fig:memory-map} показана карта виртуального адресного
пространства процесса в операционной системе Linux, выполняющейся на
процессоре архитектуры IA-32.

% http://duartes.org/gustavo/blog/post/anatomy-of-a-program-in-memory
\begin{figure}
  \newcommand{\memsection}[4]{%
    \bytefieldsetup{bitheight=#3\baselineskip}%
    \bitbox[]{10}{%
      \texttt{#1}%
      \\
      \vspace{#3\baselineskip}
      \vspace{-2\baselineskip}
      \vspace{-#3pt}
      \texttt{#2}%
    }%
    \bitbox{28}{#4}%
  }
  
  \begin{bytefield}{48}
    \begin{rightwordgroup}{1 Гбайт}
      \memsection{ffff ffff}{c000 0000}{6}{Память ядра операционной системы}
    \end{rightwordgroup}\\
    \begin{rightwordgroup}{3 Гбайт}
      \memsection{         }{         }{3}{Стек}\\
      \memsection{         }{         }{3}{↓\\\vspace{\baselineskip}↑}\\
      \memsection{\vphantom{0000 0000}}{4000 0000}{3}{Библиотеки и отображения в память}\\
      \memsection{         }{         }{2}{\vspace{\baselineskip}↑}\\
      \memsection{         }{         }{3}{Куча}\\
      \memsection{         }{         }{3}{Неинициализированные данные}\\
      \memsection{         }{         }{3}{Сегмент данных}\\
      \memsection{\vphantom{0000 0000}}{0804 8000}{3}{Сегмент текста}\\
      \memsection{0804 7fff}{0000 0000}{3}{}
    \end{rightwordgroup}\par
  \end{bytefield}
  
  \caption{Виртуальное адресное пространство процесса\label{fig:memory-map}}
\end{figure}

\Par{Стек}

\begin{figure}
  \begin{centering}
    \begin{tikzpicture}
      \usetikzlibrary{arrows}
      \pgfsetlinewidth{1pt}
      
      % «Стакан»
      \draw (0cm, 4cm) --(0cm, 0cm) --(3cm, 0cm) --(3cm, 4cm);

      % Элементы
      \foreach \y / \n in {0.2cm / 1, 1.2cm / 2, 2.2cm / 3}
      \draw (0.2cm, \y) rectangle  ++(2.6cm, 0.8cm)
      node at ++(-1.3cm, -0.4cm) {Значение \n};

      % Stack Pointer
      \draw[<-] (0cm, 2.6cm) -- ++(-0.5cm, 0cm)
      node[left] {SP};
      
      % Push и Pop
      \draw[<-] (1cm, 3.5cm) to[out=90, in=0] ++(-1cm, 1cm)
      node[left] {Push};

      \draw[->] (2cm, 3.5cm) to[out=90, in=180] ++(1cm, 1cm)
      node[right] {Pop};

    \end{tikzpicture}\par
  \end{centering}
  
  \caption{Организация стека\label{fig:stack}}
\end{figure}

% http://ru.wikipedia.org/wiki/%D0%A1%D1%82%D0%B5%D0%BA%D0%BE%D0%B2%D1%8B%D0%B9_%D0%BA%D0%B0%D0%B4%D1%80

\Par{Куча}
