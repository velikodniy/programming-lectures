\ChapterS{Введение}

\epigraph{Прошу вас, ради всего святого, сначала научитесь простому, а
потом переходите к сложному.}{Эпиктет}

Это пособие представляет собой краткое изложение вводного курса лекций
по программированию. Само по себе оно не может служить учебником, но
является кратким конспектом, помогающим вспомнить важные определения.

Первая часть посвящена основным понятиям, связанным с
программированием, а также элементам структурного
программирования. Изложение ведётся на примере языка C\#.

В первой главе вводится само понятие программирования как описания
решения какой-либо задачи. Приводятся различные способы такого
описания, рассматриваются их преимущества и недостатки.

Во второй главе рассматривается процесс преобразования программы
(понятной человеку) в машинные команды (понятные компьютеру).

Третья глава посвящена одним из важнейших понятий — типам данных и
переменным. Даются конкретные примеры на языке программирования
C\#. Также кратко рассматриваются некоторые аспекты хранения данных в
памяти компьютера.

Четвёртая глава описывает основные средства управления выполнением
программы. Материал этой темы позволит писать программы, решающие
практические задачи на языке C\#.

В конце приводится список литературы, более подробно освещающей
рассмотренные темы.

Пособие не является справочником по C\#. Если необходимо выяснить
какие-то специфические особенности этого языка, то лучше обратиться к
специальным изданиям (например, к \cite{albahari}).

Исходные тексты последней версии курса лекций находятся на
веб-странице пособия —
\url{https://github.com/velikodniy/programming-lectures}. О найденных
опечатках или неточностях можно сообщить по адресу электронной почты
\href{mailto:vadim@veikodniy.name}{vadim@velikodniy.name}.
